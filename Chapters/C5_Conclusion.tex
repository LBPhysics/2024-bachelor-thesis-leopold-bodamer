\chapter{Conclusion}
\label{Chapter5}
%----------------------------------------------------------------------------------------
%	SECTION 1
%----------------------------------------------------------------------------------------
This thesis explored the feasibility of directional routing of excitations in atomic systems using subradiant states.
Bottarelli's abstract quantum router model was adapted to an atomic system.
In this adaptation, a phase $ \theta $ was found that resembles the phase in the loop of Bottarelli's work.
It depends on atomic distances and dipole orientations.
However, this phase proved to be insufficient for achieving directional routing.
This was due to the non-Hermitian nature of the Hamiltonian governing the complex open system dynamics, which does not exhibit chirality.
The challenges of controlling interactions in fully connected setups were also discussed.
Consequently, it was necessary to modify Bottarelli's original approach to achieve effective routing.

\noindent
By changing both the atomic distance and dipole orientations, three different triangular topologies were investigated:
The equilateral triangle with aligned dipoles,
the equilateral triangle with unique dipoles, and the isosceles triangle with aligned dipoles.
It was shown that controlling the setup allows for a "brute-force"
directional routing of excitations in the atomic system by minimizing specific dipole-dipole couplings in the triangle.

\noindent
The topology with unique dipole orientations showed a very high transmission coefficient of $ T \approx 90 $ for very closely spaced atoms $d_\text{ext} = d = 0.05 \lambda$.
But this setup is hard to prepare experimentally as one would need an inhomogeneous external field applied to the apparatus.
The routing for the optimal distances around $ d_\text{ext} = 0.1 \lambda $ was demonstrated to be most effective
when the initial wave packet is centered at $k_\text{s} \approx \pi / (2 d_{\text{ext}})$ where a balance between low dissipation,
robust and fast propagation was achieved.
The equilateral triangle with aligned dipoles emerges as the most viable solution for stable transmission of $ T \approx 78 $ in experimental setups,
fitting with Bottarelli's original topology.

\noindent
For larger atomic systems, routing effectiveness improved.
The results also suggest that an isosceles triangle is suited for faster readout,
though this comes at the cost of higher leakage into the unwanted chain.


\noindent
In summary, this work provided a framework for achieving directional routing in atomic systems.
Future research could investigate the implementation of these findings in experimental conditions.
It would also be interesting to see whether similar directional routing can be achieved in systems with multiexcitation states.